% -*- coding: utf-8 -*-
%
% do 
%    fc-list | egrep Minion
% to get names of Minion fonts installed on the system
% 
% do
%    otfinfo --postscript-name MinionPro-Regular.otf
% to get PostScript name of the font
% do 
%    otfinfo --features MinionPro-Regular.otf
% to list features present in the font
% (lcdf-typetools contains otfinfo and other useful tools)

\magnification 1800
\nopagenumbers

\font\rmMinion "Minion Pro/Regular:mapping=mex-text,+onum"
\font\itMinion "Minion Pro/Italic"
\font\bfMinion "Minion Pro/Bold"
\font\biMinion "URW Bookman L/R"

\let\rm=\rmMinion
\let\it=\itMinion
\let\bf=\bfMinion
\let\bi=\biMinion

%\font\titleFont "Comic Sans MS/Bold" at 14pt
\font\titleFont "Minion Pro/Bold"
\def\title#1{\bigbreak{\special{color rgb 0 0 0.2}\titleFont#1\special{color rgb 0 0 0}}\bigskip}

\raggedright
\raggedbottom
\parindent 0pt
\parskip 6pt

\rm

\input multido

% --------

\title{Polish Diacritics}

\halign{\quad#\hfil&\quad#\hfil&\quad\hfil#\hfil&\qquad#\hfil&\quad#\hfil&\quad#\hfil\cr
  \it Name& \it U+0000&&\it Name& \it U+0000&\cr\noalign{\smallskip\hrule\smallskip}
  Aogonek& 0104&Ą&  aogonek&0105&ą\cr
  Cacute&  0106&Ć&  cacute& 0107&ć\cr
  Eogonek& 0118&Ę&  eogonek&0119&ę\cr
  Lslash&  0141&Ł&  lslash& 0142&ł\cr
  Nacute&  0143&Ń&  Nacute& 0144&ń\cr
  Oacute&  00D3&Ó&  oacute& 00F3&ó\cr
  Sacute&  015A&Ś&  sacute& 015B&ś\cr
  Zacute&  0179&Ź&  zacute& 017A&ź\cr
  Zdotaccent&017B&Ż&  zdotaccent&017C&ż\cr}

Roman: {\rm AaBbCc}  

Italic: {\it AaBbCc}

Bold: {\bf AaBbCc}

Bold Italic: {\bi AaBbCc}

\fontname\rm

\fontname\it

\fontname\bf

\fontname\bi


\title{Plain macros} 

\dots 

\ss \AA \aa \AE \ae \OE \oe \O \o

\dag \ddag \S \P -- ---

,,Polish quotation marks''  

<<guillemet left and guillemet right>>


\title{Some Polish text} 

{\bf Donald Knuth} swoimi książkami stworzył podstawy nowej dziedziny, zwanej
,,typografią cyfrową'' (ang. {\it digital typography\/}),
wypełnionej pudełkami, wymiarami, klejami (które mogą się kurczyć
i~rozciągać; jeśli potrzeba, to nawet nieskończenie).
Są tu nawet kary, marność i~wadliwość.


\title{Some cyrillic letters} 

\special{color rgb 1.0 0 0} явертюиопасдфгхйклзхцвбнмд№



\special{color rgb 0 0 0}

\title{Ornaments}

{\special{color rgb 0.3 0.3 0}

\multido{\i=57525+1}{13}{\char\i }

\multido{\i=57540+1}{11}{\char\i }

\multido{\i=57552+1}{5}{\char\i }

\char57558

}

{\font \ornmM "MinionPro-Regular:+ornm"
\ornmM qwertyuioplkjhgfdsazxcvbnm 1234567890
}

{\font \ornmSS "MinionPro-Regular:+ss01"
\ornmSS qwertyuioplkjhgfdsazxcvbnm
}

\break


\title{All Minion Features}

\halign{\quad#\hfil\qquad&\qquad#\hfil\cr
aalt&   Access All Alternates\cr
c2sc&   Small Capitals From Capitals\cr
case&   Case-Sensitive Forms\cr
cpsp&   Capital Spacing\cr
dlig&   Discretionary Ligatures\cr
dnom&   Denominators\cr
fina&   Terminal Forms\cr
frac&   Fractions\cr
hist&   Historical Forms\cr
kern&   Kerning\cr
liga&   Standard Ligatures\cr
lnum&   Lining Figures\cr
numr&   Numerators\cr
onum&   Oldstyle Figures\cr
ordn&   Ordinals\cr
ornm&   Ornaments\cr
pnum&   Proportional Figures\cr
salt&   Stylistic Alternates\cr
sinf&   Scientific Inferiors\cr
size&   Optical Size\cr
smcp&   Small Capitals\cr
ss01&   Stylistic Set 1\cr
ss02&   Stylistic Set 2\cr
sups&   Superscript\cr
tnum&   Tabular Figures\cr
zero&   Slashed Zero\cr}

\end

\font\rmMinion "MinionPro-Regular:mapping=mex-text,+onum"
\font\itMinion "MinionPro-It::+onum"
\font\bfMinion "MinionPro-Bold::+onum"
\font\biMinion "MinionPro-BoldIt::+onum"
