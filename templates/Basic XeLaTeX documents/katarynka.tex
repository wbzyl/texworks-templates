%
% Bolesław Prus, „Katarynka”
%

Na ulicy Miodowej co dzień około południa można było spotkać
jegomościa w pewnym wieku, który chodził z placu Krasińskich ku ulicy
Senatorskiej. Latem nosił on wykwintne, ciemnogranatowe palto,
popielate spodnie od pierwszorzędnego krawca, buty połyskujące jak
zwierciadła — i — nieco wyszarzany cylinder.

Jegomość miał twarz rumianą, szpakowate faworytyi siwe, łagodne
oczy. Chodził pochylony, trzymając ręce w kieszeniach. W dzień pogodny
nosił pod pachą laskę; w pochmurny — dźwigał jedwabny parasol
angielski.

Był zawsze głęboko zamyślony i posuwał się z wolna. Około Kapucynów
dotykał pobożnie ręką kapelusza i przechodził na drugą stronę ulicy,
ażeby zobaczyć u Pika, jak stoi barometr i termometr, potem znowu
zawracał na prawy chodnik, zatrzymywał się przed wystawą
Mieczkowskiego, oglądał fotografie Modrzejewskiej —- i szedł dalej.

W drodze ustępował każdemu, a potrącony uśmiechał się życzliwie.

Jeżeli kiedy spostrzegał ładną kobietę, zakładał binokle, aby
przypatrzeć się jej. Ale że robił to flegmatycznie, więc zwykle
spotykał go zawód.

Ten jegomość był to — pan Tomasz.

Pan Tomasz trzydzieści lat chodził ulicą Miodową i nieraz myślał, że
się na niej wiele rzeczy zmieniło. Toż samo ulica Miodowa pomyśleć by
mogła o nim.

Gdy był jeszcze obrońcą, biegał tak prędko, że nie uciekłaby przed nim
żadna szwaczka wracająca z magazynu do domu. Był wesoły, rozmowny,
trzymał się prosto, miał czuprynę i nosił wąsy zakręcone ostro do
góry. Już wówczas sztuki piękne robiły na nim wrażenie, ale czasu im
nie poświęcał, bo szalał — za kobietami. Co prawda miał do nich
szczęście i nieustannie był swatany. Ale cóż z tego, kiedy pan Tomasz
nie mógł nigdy znaleźć ani jednej chwili na oświadczyny, będąc zajęty
jeżeli nie praktyką, to — schadzkami. Od Frani szedł do sądu, z sądu
biegł do Zosi, którą nad wieczorem opuszczał, ażeby z Józią i Filką
zjeść kolację.

Gdy został mecenasem, czoło, skutkiem natężonej pracy umysłowej,
urosło mu aż do ciemienia, a na wąsach pokazało się kilka srebrnych
włosów. Pan Tomasz pozbył się już wówczas młodzieńczej gorączki, miał
majątek i ustaloną opinię znawcy sztuk pięknych. A że kobiety wciąż
kochał, więc począł myśleć o małżeństwie. Najął nawet mieszkanie z
sześciu pokojów złożone, urządził w nim na własny koszt posadzki,
sprawił obicia, piękne meble — i szukał żony.

Ale człowiekowi dojrzałemu trudno zrobić wybór. Ta była za młoda, a
tamtą uwielbiał już zbyt długo. Trzecia miała wdzięki i wiek właściwy,
ale nieodpowiedni temperament, a czwarta posiadała wdzięki, wiek i
temperament należyty, ale… nie czekając na oświadczyny mecenasa wyszła
za doktora…

Pan Tomasz jednak nie martwił się, ponieważ panien nie
brakło. Ekwipował się powoli, coraz usilniej dbając o to, ażeby każdy
szczegół jego mieszkania posiadał wartość artystyczną. Zmieniał meble,
przestawiał zwierciadła, kupował obrazy.

Nareszcie porządki jego stały się sławne. Sam nie wiedząc kiedy,
stworzył u siebie galerię sztuk pięknych, którą coraz liczniej
odwiedzali ciekawi. Że zaś był gościnny, przyjęcia robił świetne i
utrzymywał stosunki z muzykami, więc nieznacznie zorganizowały się u
niego wieczory koncertowe, które nawet damy zaszczycały swoją
obecnością.

Pan Tomasz był wszystkim rad, a widząc w zwierciadłach, że czoło
przerosło mu już ciemię i sięga w tył do białego jak śnieg
kołnierzyka, coraz częściej przypominał sobie, że bądź co bądź trzeba
się ożenić. Tym bardziej że dla kobiet wciąż czuł życzliwość.

Raz, kiedy przyjmował liczniejsze niż zwykle towarzystwo, jedna z
młodych pań rozejrzawszy się po salonach zawołała:

— Co za obrazy… A jakie gładkie posadzki!… Żona pana mecenasa będzie
bardzo szczęśliwa.

— Jeżeli do szczęścia wystarczą jej gładkie posadzki— odezwał się na
to półgłosem serdeczny przyjaciel mecenasa.

W salonie zrobiło się bardzo wesoło. Pan Tomasz także uśmiechnął się,
ale od tej pory, gdy mu kto wspomniał o małżeństwie, machał niedbale
ręką, mówiąc:

— Iii!…

W tych czasach ogolił wąsy i zapuścił faworyty. O kobietach wyrażał
się zawsze z szacunkiem, a dla ich wad okazywał dużą wyrozumiałość.

Nie spodziewając się niczego od świata, bo już i praktykę porzucił,
mecenas całe spokojne uczucie swoje skierował do sztuki. Piękny obraz,
dobry koncert, nowe przedstawienie teatralne były jakby wiorstowymi
słupami na drodze jego życia. Nie zapalał się on, nie unosił, ale —
smakował.

Na koncertach wybierał miejsca odległe od estrady, ażeby słuchać
muzyki nie słysząc hałasów i nie widząc artystów. Gdy szedł do teatru,
obeznawał się wprzódy z utworem dramatycznym, ażeby bez gorączkowej
ciekawości śledzić grę aktorów. Obrazy oglądał wówczas, gdy było
najmniej widzów, i spędzał w galerii cale godziny.

Jeżeli podobało mu się coś, mówił:

— Wiecie, państwo, że jest to wcale ładne.

Należał do tych niewielu, którzy najpierw poznają się na talencie. Ale
utworów miernych nigdy nie potępiał.

— Czekajcie, może się jeszcze wyrobi! — mówił, gdy inni ganili
artystę.

I tak zawsze był pobłażliwy dla niedoskonałości ludzkiej, a o
występkach nie rozmawiał.

Na nieszczęście, żaden śmiertelnik nie jest wolny od jakiegoś
dziwactwa, a pan Tomasz miał także swoje. Oto — nienawidził
kataryniarzy i katarynek.

Gdy mecenas usłyszał na ulicy katarynkę, przyśpieszał kroku i na parę
godzin tracił humor. On, człowiek spokojny — zapalał się, jak był
cichy — krzyczał, a jak był łagodny — wpadał w gniew na pierwszy
odgłos katarynkowych dźwięków.

Z tej swojej słabości nie robił przed nikim tajemnicy, nawet tłumaczył
się.

— Muzyka — mówił wzburzony — stanowi najsubtelniejsze ciało ducha, w
katarynce zaś duch ten przeradza się w funkcję machiny i narzędzie
rozboju. Bo kataryniarze są po prostu rabusie!

— Zresztą — dodawał — katarynka rozdrażnia mnie, a ja mam tylko jedno
życie, którego mi nie wypada trwonić na słuchanie obrzydliwej muzyki.

Ktoś złośliwy, wiedząc o wstręcie mecenasa do grających machin,
wymyślił niesmaczny żart — i… wysłał mu pod okna dwu kataryniarzy. Pan
Tomasz zachorował z gniewu, a następnie odkrywszy sprawcę, wyzwał go
na pojedynek.

Aż sąd honorowy trzeba było zwoływać dla zapobieżenia rozlewowi krwi o
rzecz tak małą na pozór.

Dom, w którym mecenas mieszkał, przechodził kilka razy z rąk do
rąk. Rozumie się, że każdy nowy właściciel uważał za obowiązek
podwyższać wszystkim komorne, a najpierwej panu Tomaszowi. Mecenas z
rezygnacją płacił podwyżkę, ale pod tym warunkiem wyraźnie zapisanym w
umowie, że katarynki grywać w domu nie będą.

Niezależnie od kontraktowych zastrzeżeń pan Tomasz wzywał do siebie
każdego nowego stróża i przeprowadzał z nim taką mniej więcej rozmowę:

— Słuchaj no, kochanku… A jak ci na imię?…

— Kazimierz, proszę pana.

— Słuchajże, Kazimierzu! Ile razy wrócę do domu późno, a ty otworzysz
mi bramę, dostaniesz dwadzieścia groszy. Rozumiesz?…

— Rozumiem, wielmożny panie.

— A oprócz tego będziesz brał ode mnie dziesięć złotych na miesiąc,
ale wiesz za co?…

— Nie mogę wiedzieć, jaśnie panie — odpowiedział wzruszony stróż.

— Za to, ażebyś na podwórze nigdy nie puszczał katarynek. Rozumiesz?…

— Rozumiem, jaśnie wielmożny panie.

Lokal mecenasa składał się z dwu części. Cztery większe pokoje miały
okna od ulicy, dwa mniejsze — od podwórza. Paradna połowa mieszkania
przeznaczona była dla gości. W niej odbywały się rauty, przyjmowani
byli interesanci i stawali krewni albo znajomi mecenasa ze wsi. Sam
pan Tomasz ukazywał się tu rzadko i tylko dla sprawdzenia, czy
wywoskowano posadzki, czy starto kurz i nie uszkodzono mebli.

Całe zaś dnie, o ile nie przepędzał ich za domem, przesiadywał w
gabinecie od podwórza. Tam czytywał książki, pisywał listy albo
przeglądał dokumenty znajomych, którzy prosili go o radę. A gdy nie
chciał forsować wzroku, siadał na fotelu naprzeciw okna i zapaliwszy
cygaro zatapiał się w rozmyślaniach. Wiedział on, że myślenie jest
ważną funkcją życiową, której nie powinien lekceważyć człowiek dbający
o zdrowie.

Z drugiej strony podwórza, wprost okien pana Tomasza, znajdował się
lokal wynajmowany osobom mniej zamożnym. Długi czas mieszkał tu stary
urzędnik sądowy, który spadłszy z etatu przeniósł się na Pragę. Po nim
najął pokoiki krawiec; lecz że ten lubił niekiedy upijać się i
hałasować, więc wymówiono mu mieszkanie. Później sprowadziła się tu
jakaś emerytka, wiecznie kłócąca się ze swoją sługą.

Ale od św. Jana, staruszkę, już bardzo zgrzybiałą i wcale zasobną,
pomimo jej kłótliwego usposobienia, wzięli na wieś krewni, a do lokalu
sprowadziły się dwie panie z małą, może ośmioletnią dziewczynką.

Kobiety utrzymywały się z pracy. Jedna szyła, druga wyrabiała
pończochy i kaftaniki na maszynie. Młodszą z nich i przystojniejszą
dziewczynka nazywała mamą, a starszej mówiła: pani.

I u mecenasa, i u nowych lokatorów okna przez cały dzień były
otwarte. Kiedy więc pan Tomasz usiadł na swoim fotelu, doskonale mógł
widzieć, co się dzieje u jego sąsiadek.

Były tam sprzęty ubogie. Na stołach i krzesłach, na kanapie i na
komodzie leżały tkaniny przeznaczone do szycia i kłębki bawełny na
pończochy.

Z rana kobiety same zamiatały mieszkanie, a około południa najemnica
przynosiła im niezbyt obfity obiad. Zresztą każda z nich prawie nie
odstępowała od swojej turkoczącej maszyny.

Dziewczynka zwykle siedziała przy oknie. Było to dziecko z ciemnymi
włosami i ładną twarzyczką, ale blade i jakieś nieruchawe. Czasami
dziewczynka za pomocą dwu drutów wiązała pasek z bawełnianych
nici. Niekiedy bawiła się lalką, którą ubierała i rozbierała powoli,
jakby z trudnością. Czasami nie robiła nic, tylko siedząc w oknie
przysłuchiwała się czemuś.

Pan Tomasz nie widział nigdy, ażeby dziecię to śpiewało lub biegało po
pokoju, nie widział nawet uśmiechu na bledziutkich ustach i
nieruchomej twarzy.

„Dziwne dziecko!” — mówił do siebie mecenas i począł przypatrywać się
jej uważniej.

Spostrzegł raz (było to w niedzielę), że matka dała jej mały
bukiecik. Dziewczynka ożywiła się nieco. Rozkładała i układała kwiaty,
całowała je. W końcu związała na powrót w bukiecik, włożyła go w
szklankę wody i usiadłszy w swoim oknie rzekła:

— Prawda, mamo, że tu jest smutno…

Mecenas zgorszył się. Jak mogło być smutno w domu, w którym on od tylu
lat miał dobry humor!

Jednego dnia mecenas znalazł się w swoim gabinecie około czwartej. W
tej godzinie słońce stało naprzeciw mieszkania jego sąsiadek, a
świeciło i dogrzewało bardzo mocno. Pan Tomasz spojrzał na drugą
stronę podwórza i widać zobaczył coś niezwykłego, gdyż z pośpiechem
założył na nos binokle.

Oto co spostrzegł:

Mizerna dziewczynka oparłszy głowę na ręku położyła się prawie na
wznak w swoim oknie — i — szeroko otwartymi oczyma patrzyła prosto w
słońce. Na jej twarzyczce, zwykle tak nieruchomej, grały teraz jakieś
uczucia: niby radość, a niby żal…

— Ona nie widzi! — szepnął mecenas opuszczając binokle. W tej chwili
doświadczył kłucia w oczach na samą myśl, że ktoś może wpatrywać się w
słońce, które ziało żywym ogniem.

Istotnie, dziewczynka była niewidoma od dwu lat. W szóstym roku życia
zachorowała na jakąś gorączkę; przez kilka tygodni była nieprzytomna,
a następnie tak opadła z sił, że leżała jak martwa, nie poruszając się
i nic nie mówiąc.

Pojono ją winem i bulionami, więc stopniowo przychodziła do
siebie. Ale pierwszego dnia, kiedy ją posadzono na poduszce, zapytała
matki:

— Mamo, czy to jest noc?…

— Nie, moje dziecko… A dlaczego ty tak mówisz?

Ale dziewczynka nie odpowiedziała: spać się jej chciało… Tylko
nazajutrz, gdy w południe przyszedł lekarz, spytała znowu:

— Czy to jeszcze jest noc?…

Wtedy zrozumiano, że dziewczynka nie widzi. Lekarz zbadał jej oczy i
zaopiniował, że trzeba czekać.

Ale chora, im bardziej odzyskiwała siły, tym mocniej niepokoiła się
swoim kalectwem…

— Mamo, dlaczego ja mamy nie widzę?…

— Bo tobie oczki zasłoniło. Ale to przejdzie.

— Kiedy przejdzie?…

— Niedługo.

— Może jutro, proszę mamy?

— Za kilka dni, moja dziecino.

— A jak przejdzie, to niech mi mama zaraz powie. Bo mi jest bardzo
smutno!…

Mijały dnie i tygodnie w ciągłym oczekiwaniu. Dziewczynka poczęła już
wstawać z łóżeczka. Nauczyła się chodzić po pokoju omackiem; sama
ubierała się i rozbierała powoli i ostrożnie.

Ale wzrok jej nie wracał.

Jednego razu mówiła:

— Prawda, mamo, że ja mam niebieską sukienkę?…

— Nie, dziecko, masz popielatą.

— Mama ją widzi?

— Widzę, moje kochanie.

— Tak jak i w dzień?

— Tak.

— Ja także będę widziała wszystko za kilka dni?…. Nie, może za
miesiąc…

Ale ponieważ matka nie odpowiedziała jej nic, więc mówiła dalej:

— Prawda, mamo, że na dworze ciągle jest dzień?… A w ogrodzie są
drzewa, tak jak dawniej?… Czy do nas przychodzi ten biały kotek z
czarnymi łapami?… Prawda, mamo, że ja widziałam siebie w lustrze?… Nie
ma tu lustra?…

Matka podaje jej lusterko.

— Trzeba patrzeć tutaj, o tu, gdzie jest gładkie — mówiła dziewczynka
przykładając lustro do twarzy. — Nic nie widzę! — rzekła — Czy i mama
nie widzi mnie w lusterku?

— Widzę cię, moja ptaszyno.

— Jakim sposobem?… — zawołała dziewczynka żałośnie. — Przecie jeżeli
ja nie widzę siebie, to już w lustrze nie powinno być nic…

— A tamta, co jest w lustrze, czy ona mnie widzi, czy nie widzi?…

Ale matka rozpłakała się i wybiegła z pokoju.

Najmilszym zajęciem kaleki było dotykać rękoma drobnych przedmiotów i
poznawać je.

Jednego dnia przyniosła jej matka lalkę porcelanową, ładnie ubraną, za
rubla. Dziewczynka nie wypuszczała jej z rąk, dotykała jej noska, ust,
oczu, pieściła się nią.

Poszła spać bardzo późno, wciąż myśląc o swej lalce, którą ułożyła w
pudełku wysłanym watą.

W nocy zbudził matkę szmer i szept. Zerwała się z pościeli, zapaliła
świecę i zobaczyła w kąciku swoją córkę, już ubraną i bawiącą się
lalką.

— Co ty robisz, dziecino? — zawołała. — Dlaczego nie śpisz?

— Bo już przecie jest dzień, proszę mamy — odparła kaleka.

Dla niej dzień i noc zlały się w jedno i trwały zawsze…

Stopniowo pamięć wzrokowych wrażeń poczęła zacierać się w
dziewczynce. Czerwona wiśnia stała się dla niej wiśnią gładką, okrągłą
i miękką, błyszczący pieniądz był twardym i dźwięcznym krążkiem, na
którym znajdowały się jakieś znaki w płaskorzeźbie. Wiedziała, że
pokój jest większy od niej, dom większy od pokoju, ulica od domu. Ale
wszystko to jakoś — skróciło się w jej wyobraźni.

Uwaga jej skierowała się na zmysł dotyku, powonienia i słuchu. Jej
twarz i ręce nabrały takiej wrażliwości, że zbliżywszy się do ściany,
czuła o kilka cali lekki chłód. Zjawiska odległe oddziaływały na nią
tylko przez słuch. Przysłuchiwała się więc po całych dniach.

Poznawała posuwisty chód stróża, który mówił piskliwym głosem i
zamiatał podwórko. Wiedziała, kiedy jedzie z drzewem chłopski wózek
drabiniasty, kiedy — dorożka, a kiedy — kary wywożące śmiecie.

Najmniejszy szelest, zapach, oziębienie się albo rozgrzanie powietrza
nie uszło jej uwagi. Z niepojętą bystrością pochwytywała drobne te
zjawiska i wysnuwała z nich wnioski.

Raz matka zawołała służącą.

— Nie ma Janowej — rzekła kaleka siedząc jak zwykle w kąciku. — Poszła
po wodę.

— A skąd wiesz o tym? — zapytała zdziwiona matka.

— Skąd?… Przecież wiem, że brała konewkę z kuchni, potem poszła na
drugie podwórze i napompowała wody. A teraz rozmawia ze stróżem.

Istotnie zza parkanu dolatywał szmer rozmowy dwu osób, ale tak
niewyraźny, że tylko z wysiłkiem można go było usłyszeć.

Lecz nawet rozszerzona sfera zmysłów niższych nie mogła kalece
zastąpić wzroku. Dziewczynka uczuła brak wrażeń i zaczęła tęsknić.

Pozwolono jej chodzić po całym domu i to ją nieco
uspakajało. Wydeptała każdy kamień na podwórzu, dotknęła każdej rynny
i beczki. Ale największą przyjemność robiły jej — podróże do dwu
całkiem odmiennych światów: do piwnicy i na strych.

W piwnicy powietrze było chłodne, ściany wilgotne. Przygłuszony turkot
uliczny dolatywał z góry; inne odgłosy niknęły. To była noc dla
ociemniałej.

Na strychu zaś, szczególniej w okienku, działo się całkiem
inaczej. Tam hałasu było więcej niż w pokoju. Kaleka słyszała turkot
wozów z kilku ulic: tu skupiały się krzyki z całego domu. Twarz jej
owiewał ciepły wiatr. Słyszała świergot ptaków, szczekanie psów i
szelest drzew w sąsiednim ogrodzie. Tu był dla niej dzień…

Nie dość na tym. Na strychu częściej niż w pokoju świeciło słońce, a
gdy dziewczynka skierowała na nie przygasłe oczy, zdawało jej się, że
coś widzi. W wyobraźni budziły się cienie kształtów i barw, ale takie
niewyraźne i pierzchliwe, że nic przypomnieć sobie nie mogła…

W tej właśnie epoce matka połączyła się ze swoją przyjaciółką i
przeniosła się do domu, w którym mieszkał pan Tomasz. Obie kobiety
cieszyły się z nowego lokalu, ale dla niewidomej zmiana miejsca była
prawdziwym nieszczęściem.

Dziewczynka musiała siedzieć w pokoju. Na strych i do piwnicy nie
wolno było chodzić. Nie słyszała ptaków ani drzew, a na podwórzu
panowała straszna cisza. Nigdy tu nie wstępowali handlarze starzyzny
ani druciarze, ani śmieciarki. Nie puszczano bab śpiewających pieśni
pobożne ani dziada, który grał na klarnecie, ani kataryniarzy.

Jedyną jej przyjemnością było wpatrywanie się w słońce, które przecież
nie zawsze jednakowo świeciło i bardzo prędko kryło się za domami.

Dziewczynka znowu poczęła tęsknić. Zmizerniała w ciągu kilku dni, a na
jej twarzy ukazał się wyraz zniechęcenia i martwości, który tak dziwił
pana Tomasza.

Nie mogąc widzieć, kaleka chciała przynajmniej słuchać wciąż
najrozmaitszych odgłosów. A w domu było cicho…

— Biedne dziecko! — szeptał nieraz pan Tomasz przypatrując się
smutnemu maleństwu.

„Gdybym mógł dla niej coś zrobić?” — myślał widząc, że dziecko jest
coraz mizerniejsze i co dzień niknie.

Zdarzyło się w tych czasach, że jeden z przyjaciół mecenasa miał
proces i jak zwykle oddał mu do przejrzenia papiery z prośbą o
radę. Wprawdzie pan Tomasz nie stawał już w sądach, ale jako
doświadczony praktyk umiał wskazać najwłaściwszy kierunek akcji i
wybranemu przez siebie adwokatowi udzielał pożytecznych objaśnień.

Sprawa obecna była zawikłana. Pan Tomasz im więcej wczytywał się w
papiery, tym bardziej zapalał się. W emerycie ocknął się adwokat. Nie
wychodził już z mieszkania, nie sprawdzał, czy starto kurz w salonach,
tylko zamknięty w swoim gabinecie czytał dokumenty i notował.

Wieczorem stary lokaj mecenasa przyszedł z codziennym
raportem. Doniósł, że pani doktorowa wyjechała z dziećmi na letnie
mieszkanie, że zepsuł się wodociąg, że odźwierny Kazimierz zrobił
awanturę ze stójkowym i poszedł na tydzień — do kozy. Zapytał w końcu:
czy pan mecenas nie zechce widzieć się z nowo przyjętym stróżem?…

Ale mecenas, pochylony nad papierami, palił cygaro, puszczał kółka
dymu, a na wiernego sługę nawet nie spojrzał.

Na drugi dzień pan Tomasz jeszcze siedział nad aktami; około drugiej
zjadł obiad i znowu siedział. Jego rumiana twarz i szpakowate faworyty
na szafirowym tle pokojowego obicia przypominały „studia z
natury”. Matka ociemniałej dziewczynki i jej wspólniczka robiąca
pończochy na maszynie podziwiały mecenasa i mówiły, że wygląda na
czerstwego wdowca, który ma zwyczaj od rana do wieczora drzemać nad
biurkiem.

Tymczasem mecenas, choć przymykał oczy, nie drzemał wcale, tylko
rozmyślał nad sprawą.

Obywatel X w roku 1872 zapisał swemu siostrzeńcowi folwark, a w roku
1875 — synowcowi kamienicę. Synowiec twierdził, że obywatel X był
wariatem w roku 1872, a siostrzeniec dowodził, że X oszalał dopiero w
roku 1875. Zaś mąż rodzonej siostry nieboszczyka składał nie ulegające
wątpliwości świadectwa, że X i w roku 1872, i w 1875 działał jak
obłąkany, a cały swój majątek jeszcze w roku 1869, czyli w epoce
zupełnej świadomości, zapisał siostrze.

Pana Tomasza proszono o zbadanie, kiedy naprawdę X był wariatem, a
następnie o pogodzenie trzech powaśnionych stron, z których żadna nie
chciała słuchać o ustępstwach.

Gdy tak mecenas nurzał się w powikłanych kombinacjach, zdarzył się
dziwny, trudny do pojęcia wypadek.

Na podwórzu, pod samym oknem pana Tomasza, odezwała się katarynka!…

Gdyby zmarły X wstał z grobu, odzyskał przytomność i wszedł do
gabinetu, aby pomóc mecenasowi w rozwiązywaniu trudnych zagadnień, z
pewnością pan Tomasz nie doznałby takiego uczucia jak teraz, gdy
usłyszał katarynkę!…

I żeby to przynajmniej była katarynka włoska, z przyjemnymi tonami
fletowymi, dobrze zbudowana, grająca ładne kawałki! Gdzie tam! Jakby
na większą szykanę katarynka była popsuta, grała fałszywie ordynaryjne
walce i polki, a tak głośno, że szyby drżały. Na domiar złego, trąba,
od czasu do czasu odzywająca się w niej, ryczała jak wściekłe zwierzę.

Wrażenie było potężne. Mecenas osłupiał. Nie wiedział, co myśleć i co
począć. Chwilami gotów był przypuścić, że przy odczytywaniu
pośmiertnych rozporządzeń chorego na umyśle obywatela X jemu samemu
pomieszało się w głowie i że uległ halucynacjom.

Ale nie, to nie były halucynacje. To była rzeczywista katarynka, z
popsutymi piszczałkami i bardzo głośną trąbą!

W sercu mecenasa, tego wyrozumiałego, tego łagodnego człowieka,
zbudziły się dzikie instynkty. Uczuł żal do natury, że go nie
stworzyła królem dahomejskim, który ma prawo zabijać swoich poddanych,
i pomyślał, z jaką rozkoszą położyłby w tej chwili kataryniarza
trupem!

A ponieważ u ludzi tego temperamentu, co pan Tomasz, bardzo łatwo w
gniewnym uniesieniu przechodzi się od zuchwałych projektów do
najstraszniejszych czynów, więc mecenas skoczył jak tygrys do okna i
postanowił — zwymyślać kataryniarza najgorszymi wyrazami.

Już wychylił się i otworzył usta, aby krzyknąć: „Ty… próżniaku
jakiś!…” — gdy wtem usłyszał dziecięcy głos.

Spojrzał naprzeciwko.

Mała niewidoma dziewczynka tańczyła po pokoju, klaszcząc w ręce. Blada
jej twarz zarumieniła się, usta śmiały się, a pomimo to z zastygłych
oczu płynęły łzy jak grad.

Ona, biedactwo, w tym domu spokojnym dawno już nie doświadczyła tylu
wrażeń! Jak pięknym zjawiskiem wydawały się jej fałszywe tony
katarynki! Jak wspaniałym był ryk trąby, która mecenasa mało nie
przyprawiła o apopleksję.

Na dobitkę, kataryniarz, widząc uciechę dziecka, zaczął przytupywać
wielkim obcasem w bruk i od czasu do czasu pogwizdywać niby lokomotywa
przed spotkaniem się pociągów.

Boże! Jak on ślicznie gwizdał…

Do gabinetu mecenasa wpadł wierny lokaj ciągnąc za sobą stróża i
wołając:

— Ja mówiłem temu gałganowi, jaśnie panie, żeby natychmiast wygnał
kataryniarza! Mówiłem, że od jaśnie pana dostanie pensję, że my mamy
kontrakt… Ale ten cham! Tydzień temu przyjechał ze wsi i nie zna
naszych obyczajów.

— No, teraz posłuchaj — krzyczał lokaj targając za ramię oszołomionego
stróża — posłuchaj, co ci sam jaśnie pan mecenas powie!

Kataryniarz grał już trzecią sztuczkę tak fałszywie i wrzaskliwie jak
dwie pierwsze.

Niewidoma dziewczynka była upojona.

Mecenas odwrócił się do stróża i rzekł ze zwykłą sobie flegmą, choć
był trochę blady:

— Słuchaj no, kochanku… A jak ci na imię?…

— Paweł, jaśnie panie.

— Otóż, mój Pawle, będę ci płacił dziesięć złotych na miesiąc, ale
wiesz za co?…

— Za to, ażebyś na podwórze nigdy nie puszczał katarynek! — wtrącił
śpiesznie lokaj.

— Nie — rzekł pan Tomasz. — Za to, ażebyś przez jakiś czas co dzień
puszczał katarynki. Rozumiesz?

— Co pan mówi?… — zawołał służący, którego nagle rozzuchwalił ten
niepojęty rozkaz.

— Ażeby, dopóki się z nim nie rozmówię, puszczał co dzień katarynki na
podwórze — powtórzył mecenas wsadzając ręce w kieszenie.

— Nie rozumiem pana!… — odezwał się służący z oznakami obrażającego
zdziwienia.

— Głupiś, mój kochany! — rzekł mu dobrotliwie pan Tomasz.

No, idźcie do roboty — dodał.

Lokaj i stróż wyszli, a mecenas spostrzegł, że jego wierny sługa coś
towarzyszowi swemu szepcze do ucha i pokazuje palcem na czoło…

Pan Tomasz uśmiechnął się i jakby dla stwierdzenia ponurych domysłów
famulusa wyrzucił katarynce dziesiątkę.

Następnie wziął kalendarz, wyszukał w nim listę lekarzy i zapisał na
kartce adresy kilku okulistów. A że kataryniarz odwrócił się teraz do
jego okna i za jego dziesiątkę począł przytupywać i wygwizdywać
jeszcze głośniej, co już okrutnie drażniło mecenasa, więc zabrawszy
kartkę z adresami doktorów wyszedł mrucząc:

— Biedne dziecko!… Powinienem był zająć się nim od dawna…
