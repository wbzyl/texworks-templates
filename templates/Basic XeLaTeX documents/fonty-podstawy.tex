% !TEX TS-program = xelatex
% !TEX encoding = UTF-8

\documentclass[a4paper,12pt]{article}

\usepackage{fontspec,xunicode}
\defaultfontfeatures{Ligatures=TeX,Scale=MatchLowercase}
\usepackage{mathpazo}
% http://www.gust.org.pl/projects/e-foundry/tex-gyre/
\setmainfont{TeX Gyre Bonum}

\usepackage{polyglossia}
\setdefaultlanguage{polish}

\begin{document}
\pagestyle{empty}

\section*{Podstawy fontologii}

Pakiet \emph{fontspec} ułatwia wybór fontów, które zostaną
użyte przez program Xe\TeX{} (lub Lua\TeX) w~czasie składu dokumentu.

Dokument ten korzysta z polskich wzorców dzielenia wyrazów.

\bigskip

Notice the font features used to load the default fonts in the preamble.
The first, \verb|Ligatures=TeX|, enables regular \TeX{} ligatures like
\verb|,,---''| for ,,---''.

The second, \verb|Scale=MatchLowercase|, automatically scales the fonts to
the same \emph{x-height}.

\end{document}

% Projekt TeX Gyre:
%   http://www.gust.org.pl/projects/e-foundry/tex-gyre/

\setmainfont{TeX Gyre Bonum}
\setsansfont{TeX Gyre Heros}
\setmonofont{DejaVu Sans Mono}

% Zadanie: Zainstalować font Kurier:
%
%   http://jmn.pl/kurier-i-iwona/
%
%   fc-cache ~/.fonts
%   fc-list | egrep Kurier

\setmonofont{Kurier}
\setmonofont{Kurier Medium}
\setmonofont{Kurier Heavy}
